\documentclass{article}
\usepackage[utf8]{inputenc}
\title{Relatório do Projeto}
\author{Nome Completo}
\date{}

\begin{document}
\maketitle
\section{Introdução}

Descreva o \textit{framework} que você analisará e contextualize as tarefas que ele pode ser útil.
Caso o foco esteja na implementação de um algoritmo, descreva qual o algoritmo e o artigo que se baseou.

\section{Implementação}

Descreva em alto nível o algoritmo e a estruturação da sua implementação.

\section{Framework}

Descreva em detalhes sua análise do \textit{framework}, cenários que ele pode ser útil e o caso específico que você considerou para avaliá-lo.

\section{Metodologia e Resultados}

Descreva a metodologia utilizada para avaliar o algoritmo e sua implementação ou o \textit{framework} escolhido.
Apresente tabelas de resultados e gráficos que auxiliem a compreensão dos
mesmos.
Note que seu script de experimento deve deixar claro o código gerador de cada tabela/gráfico.

\section{Comentários Finais}

Descreva os principais resultados obtidos e comente sobre:
(i) dificuldades encontradas;
(ii) ideias que não foram exploradas e razões;
\textbf{Não esquecer de enviar junto ao relatório o script que executa os experimentos na base de dados e gera os resultados apresentados aqui.}

\end{document}
